% Created 2025-11-15 Sat 11:44
% Intended LaTeX compiler: pdflatex
\documentclass[mathserif]{beamer}
\usepackage[utf8]{inputenc}
\usepackage[T1]{fontenc}
\usepackage{graphicx}
\usepackage{longtable}
\usepackage{wrapfig}
\usepackage{rotating}
\usepackage[normalem]{ulem}
\usepackage{amsmath}
\usepackage{amssymb}
\usepackage{capt-of}
\usepackage{hyperref}
\hypersetup{colorlinks=true,urlcolor=blue}
\graphicspath{{./figs/}}  %% Where the figures live
\usenavigationsymbolstemplate{}
\usepackage{lmodern}
\usepackage[useregional=numeric]{datetime2}
\setbeamertemplate{footline}{ \hfill \raisebox{5pt}{ \insertframenumber \hspace{2pt}} }
\usepackage{xcolor}
\usepackage{minted}
\usemintedstyle{kauffman}
\usepackage{xpatch}
\xpatchcmd{\mintinline}{\begingroup}{\begingroup\let\itshape\relax}{}{}
\xpatchcmd{\minted}{\VerbatimEnvironment}{\VerbatimEnvironment\let\itshape\relax}{}{}
\usepackage{relsize} %% used to decrease existing font size
%% type writer font for line numbers
\renewcommand{\theFancyVerbLine}{\ttfamily \textsmaller[1]{\arabic{FancyVerbLine}}}
\usetheme{default}
\author{Chris Kauffman}
\date{\it Last Updated: \newline \input{|"/usr/bin/date"}}
\title{Technica ShellSlayer Fall 2025 Workshop}
\hypersetup{
 pdfauthor={Chris Kauffman},
 pdftitle={Technica ShellSlayer Fall 2025 Workshop},
 pdfkeywords={},
 pdfsubject={},
 pdfcreator={Emacs 30.2 (Org mode 9.7.11)}, 
 pdflang={English}}
\begin{document}

\maketitle
\begin{frame}[label={sec:orgd7cc050}]{Preamble}
Visit: \url{https://go.umd.edu/shellslayer}
\begin{itemize}
\item Includes link to github page for ShellSlayer
\item Includes a link to the \textbf{Cheat Sheet} in case you are stuck
\item Has a Feedback Form to fill in at the end of this session
\end{itemize}
\end{frame}
\begin{frame}[label={sec:org803c458}]{Welcome Slayers}
\begin{block}{Immediate Goals}
\begin{itemize}
\item Illustrate Docker as a way to get a shell environment / run programs
\item Learn a some useful Unix commands and shell tricks that may improve
your life on the command line
\item Have some fun!
\end{itemize}
\end{block}
\begin{block}{Long Term Goals}
\begin{itemize}
\item Consider contributing to the tutorial via playtesting, writing
levels, etc.
\item Add some levels on some topics not covered
\end{itemize}
\end{block}
\begin{block}{Thank yous}
\begin{itemize}
\item UMD ACM for Inspiring the initial version of Shell Slayer
\item Technica for the invitation to share here
\item All of you for choosing to show up!
\end{itemize}
\end{block}
\end{frame}
\begin{frame}[label={sec:org331d2aa}]{Docker Setup}
\vspace{-.4in}
\begin{columns}
\begin{column}{.75\columnwidth}
\begin{block}{}
If you've not installed Docker already, search the internet for
\textbf{``Install Docker''} and follow the instructions
\end{block}
\end{column}
\begin{column}{.25\columnwidth}
\begin{block}{}
\begin{center}
\includegraphics[width=\textwidth]{docker-logo.png}
\end{center}
\end{block}
\end{column}
\end{columns}
\begin{block}{}
\vspace{-.2in}
\begin{itemize}
\item Docker is a system by which ``containers'' can be set up to run
programs
\item Containers provide a skeleton operating system along with
libraries/utilities so that programs can run reliably
\item Similar in spirit to Virtual Machines or Emulation BUT has some
technical distinctions which you can read about on your own time
\item Docker allows for download/run of containers with programs in them
via a GUI (blegh) or command line (yay!)
\end{itemize}
\end{block}
\end{frame}
\begin{frame}[label={sec:org06eb263},fragile]{Running Shell Slayer}
 Start the ``Docker Desktop'' application (via clicking things) then use
a command shell to download / run the shellslayer image via:

\footnotesize
\begin{block}{Windows: \texttt{cmd.exe}}
\begin{minted}[numbersep=0.5em]{sh}
>> docker run -it --rm kauffman77/shellslayer
\end{minted}
\end{block}
\begin{block}{MacOS: \texttt{Terminal.app}}
\begin{minted}[numbersep=0.5em]{sh}
>> docker run -it --rm --platform=linux/amd64 kauffman77/shellslayer
\end{minted}
\end{block}
\begin{block}{Linux: Terminal Emulator of Choice}
\begin{minted}[numbersep=0.5em]{sh}
>> sudo docker run -it --rm kauffman77/shellslayer
\end{minted}
\texttt{sudo} may not be necessary depending on your system config
\end{block}
\begin{block}{Updating your Image}
If you've previously done Shell Slayer, update your Docker image via
\begin{minted}[numbersep=0.5em]{sh}
>> docker pull kauffman77/shellslayer
\end{minted}
\end{block}
\end{frame}
\begin{frame}[label={sec:orgb60e708},fragile]{Structure of the Exercise}
 \begin{description}
\item[{daemon (noun)}] \begin{enumerate}
\item Archaic spelling of demon
\item In multitasking computer operating systems, a program that runs
as a background process rather than being under the direct
control of an interactive user.
\end{enumerate}
\end{description}

\normalsize
\emph{You sense a foul presence in the process tree...}
\begin{itemize}
\item A ``Daemon'' process is present in the system and must be killed.
\item To do so, you must switch to be the Root user (administrator)
\item Becoming root requires a password
\item Each level (\texttt{level\_00\_less}, \texttt{level\_02\_editing}, ...) teaches about
a Unix tool/technique and when completed properly will reveal part
of the root password (one of the Daemon's names)
\item Complete all levels, get the whole password, and slay the
Daemon
\end{itemize}
\end{frame}
\begin{frame}[label={sec:orgfdad617},fragile]{Workshop Plan}
 \vspace{-.4in}
\begin{columns}
\begin{column}{1.1\columnwidth}
\begin{block}{}
\normalsize
\begin{center}
\begin{tabular}{lrl}
\hline
Level & Name & Overview\\
\hline
\texttt{level\_00\_less} & 1 & Viewing text files with \texttt{cat / less}\\
\texttt{level\_01\_touch} & - & \texttt{touch} to create /freshen files\\
\texttt{level\_02\_editing} & 2 & Editing text files with \texttt{vi / nano}\\
\texttt{level\_05\_grep} & 3 & Searching for text with \texttt{grep}\\
\texttt{level\_10\_sh\_loops} & 4 & Looping over files in the shell\\
\texttt{level\_15\_find} & 5 & Locating files with \texttt{find}\\
\texttt{level\_99\_su\_kill} & - & Changing to ``root'' and killing processes\\
\hline
UNUSED &  & \\
\texttt{level\_20\_sed} & 6 & Substituting text in files with \texttt{sed}\\
 &  & Find this in the full version\\
\hline
\end{tabular}
\end{center}
\normalsize

For each level...
\begin{itemize}
\item Give a short overview
\item Give participants a few minutes to try commands, solve the exercise
\item Regroup to discuss the stage solution and fill in a Daemon name
\end{itemize}
\end{block}
\end{column}
\end{columns}
\end{frame}
\begin{frame}[label={sec:org18b2b41},fragile]{Thank You for Your Participation}
 \textbf{Participant Feedback at \url{https://go.umd.edu/shellslayer}}


\vspace{-.2in}
\begin{columns}
\begin{column}{.5\columnwidth}
\begin{block}{}
\begin{itemize}
\item If you enjoyed the workshop and have feedback, provide it in the
survey
\item If you are interested in contributing to the tutorial, email me 
\begin{itemize}
\item Chris Kaufman
\item UMD Computer Science
\item \texttt{<profk@umd.edu>}
\end{itemize}
\end{itemize}
\end{block}
\end{column}
\begin{column}{.5\columnwidth}
\begin{block}{}
\begin{center}
\includegraphics[width=\textwidth]{witcher-geralt.jpg}
\end{center}
\emph{The world doesn't need a hero.}

\emph{It needs a professional.}
\end{block}
\end{column}
\end{columns}
\end{frame}
\end{document}
